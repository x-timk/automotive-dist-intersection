\documentclass{memoir}
\usepackage[T1]{fontenc}
\usepackage[utf8]{inputenc}
\usepackage[italian,english]{babel}
\usepackage{tikz}
\usetikzlibrary{automata,positioning,topaths}

\title{Implementazione di un sistema di incrocio stradale intelligente \\ for the Project of Distributed Systems}

\author{Tiziano Mele\\Valentino Picotti\\DMIF, University of Udine, Italy}

\date{Version 1, \today}

\begin{document}


%\begin{titlingpage}
\maketitle
\begin{abstract}
  Lo scopo del presente documento è quello di descrivere una possibile soluzione
  distribuita per il problema degli incroci stradali intelligenti.
\end{abstract}
%\end{titlingpage}

\chapter{Introduction}\label{ch:intro}

\paragraph{Descrizione dell'applicazione}


Un sistema di incrocio stradale intelligente vuole essere una soluzione al
problema della congestione stradale. Con l'avvento dei sistemi wireless ci si
chiede se sia possibile per i veicoli stradali di poter decidere direttamente
tra di loro chi debba impegnare l'incrocio, eliminando quindi i semafori. La
soluzione proposta per il problema vuole soddisfare i seguenti requisiti:
\begin{itemize}
\item Sicurezza stradale: Le auto che attraversano l'incrocio non devono scontrarsi
\item Fairness: Ogni macchina che si trova ad un incrocio deve prima o poi attraversarlo
\item Liveness: Non possono verificarsi situazioni in cui nessuna macchina
  nell'incrocio può muoversi
\item Fault Tolerant: Il sistema deve funzionare in caso di guasti fisici o
  logici. Un guasto fisico comporta che la macchina è attiva nella rete ma non
  può muoversi. Un guasto logico comporta che la macchina non è più
  raggiungibile nella rete, ma comunque presente fisicamente nell'ambiente.
\item Distribuito: Le macchine si parlano tra di loro senza un server centrale
  che potrebbe diventare un SPOF.
\item Modulare (Scaling Transparency): E' possibile utilizzare il sistema su un
  qualsiasi tipo di incrocio: è sufficiente cambiare il grafo di descrizione
  dell'incrocio (non è necessario agire modificando la struttura del sistema e
  degli algoritmi)
\end{itemize}

Struttura complessiva implementazione: Lo spazio fisico è partizionato in una
griglia dove ogni macchina può occupare al più una cella. La griglia è poi
rappresentata da un grafo orientato e pesato in cui i nodi rappresentano le
singole celle della griglia. Si ha una collisione quando due auto vanno ad
occupare la stessa cella nella griglia (lo stesso nodo del grafo). Le macchine
sono dotate di sensori di prossimità, gps, e un modulo wifi per comunicare con
le altre macchine. Una macchina può avanzare solo se il sensore di prossimità
non rileva ostacoli e mediante il modulo gps la macchina è in grado di capire se
si trova in prossimità di un incrocio. In quest'ultimo caso prima di avanzare
deve coordinarsi con eventuali altre macchine presenti. Il coordinamento si
traduce in una fase di elezione di un leader dell'incrocio, che poi deciderà
quali macchine far passare.

\paragraph{Aspetti distribuiti del sistema}

\begin{itemize}
\item Ogni macchina è un'entità indipendente con conoscenza parziale
  dell'ambiente circostante (data dai suoi sensori).
\item Le macchine si scoprono e parlano tra di loro autonomamente senza bisogno
  di un server centrale.
\item Fase di elezione del leader dell'incrocio. L'algoritmo scelto per
  l'elezione è il bully.
\end{itemize}

\paragraph{Trasparenze del sistema}

\begin{itemize}
\item Access transparency: Non c'è un concetto di risorsa remota a cui le
  macchine accedono. Le macchine utilizzano dati ricavati da sensori oppure
  messaggi ricevuti da altre macchine.
\item Location transparency: Una macchina comunica con i vicini senza conoscerne
  la posizione fisica nell'ambiente.
\item Concurrency transparency: Più macchine possono impegnare l'incrocio senza
  interferire l'una con l'altra.
\item Replication transparency: non pertinente con il problema.
\item Failure transparency: sono ammessi fallimenti fisici e logici delle
  macchine. Se la macchina si guasta fisicamente è lei stessa a "chiamare il
  carro attrezzi" e quindi a rimuoversi dall'ambiente. Se la macchina si guasta
  logicamente, è la macchina dietro che passato un certo timeout chiede
  l'intervento di un carro attrezzi.
\item Mobility transparency: non pertinente
\item Performance transparency: non pertinente
\item Scaling transparency: non è necessario modificare la struttura e gli
  algoritmi del sistema al crescere delle macchine.
\end{itemize}

% In this chapter you describe the main problem, and an idea of the solution. It
% is not necessary to be very detailed or formal, but it is important to explain
% which are the main aims and issues from the point of view of Distributed
% Systems:
% \begin{itemize}
% \item A description of the application.
% \item The overall structure of the implementation: how resources are deployed, which are the players, the r\^oles.
% \item The distributed system features (and the transparencies) and algorithms you intend to implement.
% \item Your plan for testing the system.
% \item A schedule for how you plan to carry our your design and implementation
% \end{itemize}

\chapter{Analysis}\label{ch:analysis}

Come primo passaggio viene creato un processo "environment" dove viene salvato
lo stato globale dell'incrocio. L'incrocio viene tradotto in un grafo orientato
pesato, dove ogni nodo conterrà le seguenti informazioni:
\begin{itemize}
\item Nome
\item Tipologia di nodo: TailNode(Nodo in coda) TopNode (Nodo in prossimità di
  un incrocio) CrossNode(Nodo incrocio)
\item Elenco di macchine che occupano il nodo stesso
\end{itemize}
Il peso degli archi del grafo viene assegnato in base alla distanza fisica tra i
due nodi.
Questo modulo viene interrogato dalle varie macchine per simulare dati relativi
alla sensoristica.

Ogni veicolo sarà implementato come un processo che quindi potrà interrogare i
sensori per ottenere le informazioni su posizione, ostacoli, percorso più breve
verso la sua destinazione.

Le macchine sono in relazione client-server con l'environment. Ogni automobile è
una macchina a stati finiti per la quale si prevede di utilizzare il behaviour
di erlang gen\_statem. Gli stati principali per un veicolo sono:
\begin{itemize}
\item \texttt{INQUEQUE}: La macchina si trova in questo stato quando è in coda
  (non è la prima macchina della fila)
\item \texttt{DISCOVER}: La macchina è la prima della file. In questo stato la
  macchina manda un broadcast per scoprire le auto vicine.
\item \texttt{ELECTION}: La macchina è la prima della fila. E' in corso
  l'elezione di un leader dell'incrocio
\item \texttt{SLAVE}: L'elezione si è conclusa e la macchina non è leader. Mando
  il mio percorso al leader e attendo risposta "verde"/"rosso".
\item \texttt{M-FETCH}: La macchina è stata eletta leader dell'incrocio. Attendo
  i percorsi delle macchine SLAVE.
\item \texttt{M-COORD}: In base ai percorsi e alle priorità delle macchine mando
  messaggi "verde"/"rosso" agli SLAVE.
\item \texttt{CROSSING}: La macchina è in questo stato se si trova su un nodo
  del grafo di tipo CrossNode.
\item \texttt{PH-FAULT}: La macchina ha un guasto fisico.
\end{itemize}

Ogni veicolo può ricevere uno dei seguenti eventi:
\begin{itemize}
\item \texttt{disc}: La macchina è appena arrivata in prossimità dell'incrocio e
  manda in broadcast un "discover" con il suo pid e il suo percorso. A livello
  pratico per simulare il broadcast il messaggio di discover viene mandato
  all'environment che poi rigira alle altre macchine. Queste quindi a loro volta
  mandano un messaggio di "hellofrom" alla nuova macchina (con il loro
  percorso).
\item \texttt{disc\_tm}: lo stato \texttt{DISCOVER} ha un suo timeout. Una volta
  trascorso viene innescato questo evento.
\item \texttt{wait(tag:*)}: se la macchina si trova in stato \texttt{DISCOVER} e
  riceve questo evento rimane nello stato \texttt{DISCOVER} e resetta il
  timeout.
\item \texttt{hello(pid)}: se la macchina si trova in stato \texttt{DISCOVER},
  mi salvo il pid ricevuto tra i conosciuti.
\item \texttt{elect}: Messaggio di \emph{election} dell'algoritmo bully.
\item \texttt{coord}: Messaggio di \emph{coordinator} dell'algoritmo bully.
\item \texttt{ans}: Messaggio di \emph{answer} dell'algoritmo bully.
\item \texttt{elect\_tm}: Evento di timeout (elezione fallita) dell'algoritmo bully.
\item \texttt{route}: messaggio contenente id, priorità, lista macchine con cui
  è in conflitto.
\item \texttt{route\_tm}: evento di timeout per lo stato \texttt{M-FETCH}.
\item \texttt{green}: Messaggio che notifica a macchina in stato \texttt{SLAVE}
  che può attraversare incrocio.
\item \texttt{red}: Messaggio che notifica a macchina in stato \texttt{SLAVE}
  che non può attraversare incrocio.
\item \texttt{gr\_timeout}: evento di timeout per lo stato \texttt{SLAVE} e
  \texttt{M-COORD}.
\end{itemize}

\begin{tikzpicture}[shorten >=1pt,
  node distance=2.5cm,on grid,auto,
  bend angle=30, font=\scriptsize\ttfamily, scale=0.3]
  \node[state,initial,accepting] (queue)                 {INQUEUE};
  \node[state,accepting] (disc) [right=of queue] {DISCOVER};
  \node[state,accepting] (elec) [right=of disc,xshift=0.5cm]  {ELECTION};
  \node[state,accepting] (slave) [xshift=1cm,above right=of elec]  {SLAVE};
  \node[state,accepting] (m-fetch) [xshift=1cm,below right=of elec]  {M-FETCH};
  \node[state,accepting] (m-coord) [right=of m-fetch,xshift=0.5cm]  {M-COORD};
  \node[state,accepting] (cross) [above right=of m-coord]  {CROSSING};
  \node[state,accepting,initial,initial text=fault] (fault) [below=of disc]  {PH-FAULT};
  % \node[state,accepting]         (q_1) [right=of q_0] {$q_1$};
  % \node[state,accepting]         (q_2) [right=of q_1] {$q_2$};
  \path[->] (queue)   edge node {on\_top} (disc)
            (disc)    edge [loop above] node [xshift=-0.5cm] {disc,hello,wait} ()
                      edge node {disc\_tm} (elec)
                        edge [bend right] node[yshift=-0.1cm] {elec} (elec)
                        edge [bend left] node[yshift=0.1cm,xshift=0.7cm] {coord} (slave)
              (elec)    edge [loop below] node {disc,elect,ans} ()
                        edge node {coord} (slave)
                        edge [bend right] node[yshift=0.45cm] {elec\_tm} (disc)
                        edge node[xshift=-0.4cm,yshift=0.2cm] {master} (m-fetch)
              (slave)   edge [loop above] node {disc} ()
                        edge node {green} (cross)
                        edge [bend right=50] node[xshift=-0.5cm,yshift=0.7cm] {red,gr\_tm} (disc)
              (m-fetch) edge [loop above] node {route,disc} ()
                        edge node {route\_tm} (m-coord)
              (m-coord) edge [loop above] node {disc,route} ()
                        edge [below] node [xshift=0.5cm] {green} (cross)
                        edge [bend left=50] node {red,gr\_tm} (disc)
              (cross)   edge [loop above] node {disc} ();
\end{tikzpicture}

Ogni veicolo ha un suo valore di priorità iniziale, che viene incrementato ogni
volta che il veicolo riceve il messaggio \texttt{red} (lo può ricevere solo se è
in stato \texttt{SLAVE} o \texttt{M-COORD}). Il leader dell'incrocio decide chi
far passare nel seguente modo:
\begin{itemize}
\item Ordina le macchine per priorità;
\item Farà sicuramente passare la prima.
\item Calcola un insieme massimale di macchine indipendenti rispetto alla prima
\end{itemize}

% In this chapter, we describe in detail functional and non-functional
% requirements of a solution for the problem. Ogni macchina ha un suo "semaforo"
% interno: quando si trova in prossimità di un incrocio può passare solo se il suo
% semaforo è verde

% \section{Functional requirements}
% Which functions must be offered to users / other programs?  Which are the input data and the output data? Which is the expected effect? 

% \section{Non functional requirements}
% Everything about mode and transparencies: availability, mobility, security, fault tolerance, etc.

% Are there execution time bounds? Minimum data rates?

% If requested, specific platforms/languages/middlewares requirements for the implementation can be decided here. (E.g.: if the project is on a SOA, we may request that functions are offered via SOAP or RESTful services). 



% \chapter{Project}

% This chapter is devoted to the description of the general architectures, and specific algorithms.

% \section{Logical architecture}
% Describe the components of your systems: modules/objects/components/services.
% For each component, describe the functionalities it implements, and by who is used.

% \section{Protocols and algorithms}
% Communication between components.  UML sequence diagrams go here.

% Also, put here a detailed description of distributed algorithms used to solve specific problems of the project.

% \section{Physical architecture and deployment}
% Which nodes and platforms involved, and where each component is deployed.

% \section{Development plan}
% Since it is difficult to predict just how hard implementing a new system will be, you should formulate as a set of ``tiers,'' where the basic tier is something you?re sure you can complete, and the additional tiers add more features, at both the application and the system level.

% \chapter{Implementation}

% Details about the implementation: every choice about platforms, languages, software/hardware, middlewares, which has not been decided in the requirements.


% Important choices about implementation should be described here; e.g., peculiar data structures.


% \chapter{Validation}

% Check if requirements from Chapter~\ref{ch:analysis} have been fulfilled.
% Quantitative tests (simulations) and screenshots of the interfaces are put here.


% \chapter{Conclusions}

% What has been done with respect to what has been promised in Chapters~\ref{ch:intro} and \ref{ch:analysis}, and what is left out.

% \appendix

% \chapter{Appendix}

% In the Appendix you can put code snippets, snapshots, installation instructions, etc.


% \chapter*{Evaluation}
% Your system will be judged mainly on how it operates as a distributed system. The primary evaluation will be according to whether your system has the following attributes:
% \begin{itemize}
% \item  It should be an interesting distributed system, making use of some of the algorithms we have covered in class for distributed synchronization, replication, fault tolerance and recovery, security, etc.
% \item The software should be well designed and well implemented, in terms of the overall architecture and the detailed realization.
% \item You should devise and apply systematic testing procedures, at both the unit and systems levels.
% \item The system should operate reliably and with good performance, even in the face of failures.
% \end{itemize}
% Important, but secondary considerations include:
% \begin{itemize}
% \item Time taken to do the project (the sooner the better, but do not miss details in order to end sooner)
% \item  How nice is the application's appearance: does it have a nice interface or a compelling visual display?
% \end{itemize}

\end{document}


% Local Variables:
% ispell-dictionary: italiano
% End:
