\documentclass{memoir}
\usepackage[T1]{fontenc}
\usepackage[utf8]{inputenc}
\usepackage[english,italian]{babel}
\usepackage[italian]{cleveref}
\usepackage{tikz}
\usetikzlibrary{automata,positioning,topaths}

\title{Implementazione di un sistema di incrocio stradale intelligente \\ for the Project of Distributed Systems}

\author{Tiziano Mele\\Valentino Picotti\\DMIF, University of Udine, Italy}

\date{Version 1, \today}

\begin{document}


%\begin{titlingpage}
\maketitle
\begin{abstract}
  Un sistema di incrocio stradale intelligente vuole essere una soluzione al
  problema della congestione stradale. Con l'avvento dei sistemi wireless e
  autonomici ci si chiede se sia possibile per i veicoli stradali poter decidere
  direttamente tra di loro chi debba impegnare l'incrocio. Di seguito si indaga
  una soluzione che permatta di eliminare le lanterne semaforiche in modo da
  ridurre i tempi medi di attesa dei veicoli in coda.
\end{abstract}
%\end{titlingpage}

\chapter{Introduzione}\label{ch:intro}

La soluzione che viene proposta in questo lavoro vuole essere di tipo
distribuito, dove i singoli veicoli sono entità indipendenti e con conoscenza
parziale dell'ambiente circostante data dalla sensoristica di bordo. Compito dei
veicoli è quello di formare una rete peer-to-peer che permetta loro di
comunicare autonomamente, senza bisogno di un server centrale.

% TODO: Da riconsiderare per capitolo 3.1
% Inoltre la soluzione mira ad essere fisicamente plausibile, ipotizzando un
% deploy in una situazione reale (o mondo reale)
% soluzione fisicamente plausibile
% al funzionamento fisico della sensoristica
% Molte scelte sono state effettuate pensando al livello tecnologico della
% sensoristica ad oggi disponibile; ad esempio è stato tenuto conto del fatto che
% i dispositivi wireless hanno un raggio di azione limitato.

I principali requisiti che si vogliono soddisfare sono:
\begin{description}
\item[Sicurezza stradale] Le auto che attraversano l'incrocio non devono
  scontrarsi.
\item[Liveness] Ogni macchina che si trova ad un incrocio deve prima o poi
  attraversarlo.
\item[Fairness] Quando un'auto chiede di impegnare l'incrocio, esiste un tempo
  limite entro cui le viene assegnato.
\item[Deadlock freedom] Non possono verificarsi situazioni in cui nessuna
  macchina nell'incrocio può muoversi.
\item [Fault Tolerant] Il sistema deve funzionare in caso di guasti fisici o
  logici. Un guasto fisico comporta che la macchina è attiva nella rete ma non
  può muoversi. Un guasto logico comporta che la macchina non è più
  raggiungibile nella rete, ma comunque presente fisicamente nell'ambiente.
\item [Distribuito] Le macchine si parlano tra di loro senza un server centrale
  che potrebbe diventare un SPOF.
\item [Generale] E' possibile utilizzare il sistema su un qualsiasi tipo di
  incrocio: è sufficiente cambiare il grafo di descrizione dell'incrocio (non è
  necessario agire modificando la struttura del sistema e degli algoritmi).
\end{description}

\subsection{Struttura complessiva dell'implementazione}

Lo spazio fisico è partizionato in una griglia dove ogni macchina può occupare
al più una cella. La griglia è poi rappresentata da un grafo orientato e pesato
in cui i nodi rappresentano le singole celle della griglia. Si ha una collisione
quando due auto vanno ad occupare la stessa cella nella griglia (lo stesso nodo
del grafo). Le macchine sono dotate di sensori di prossimità, gps, e un modulo
wifi per comunicare con le altre macchine. Una macchina può avanzare solo se il
sensore di prossimità non rileva ostacoli e mediante il modulo gps la macchina è
in grado di capire se si trova in prossimità di un incrocio. In quest'ultimo
caso prima di avanzare deve coordinarsi con eventuali altre macchine presenti.
Il coordinamento si traduce in una fase di elezione di un leader dell'incrocio,
che poi deciderà quali macchine far passare.


Per testare il sistema è stato implementato un ambiente simulato le cui
funzionalità rispecchino il più possibile quelle di un ambiente reale.

\chapter{Analisi}\label{ch:analysis}
In questo capitolo si descrivono requisi funzionali e non funzionali della
soluzione fornita.

\section{Requisiti funzionali}

Al fine di poter fornire una simulazione realistica sono stati individuati i
seguenti requisiti funzionali per gli attori che vi partecipano.

\subsection{Veicolo}

Un veicolo deve soddisfate i seguenti requisiti funzionali:
\begin{itemize}
\item Muoversi in autonomia
\item Poter interrogare i sensori di prossimità
\item Conoscere la sua posione gps attuale e più in generale conoscere il suo
  intero tragitto
\item Riconoscere se si trova in prossimatà di un incrocio, al suo interno o se
  sta percorrendo un tratto senza incroci
\item Riconoscere eventuali macchine guaste che ostacolino il suo percorso
\item Richiedere l'intervento del carro attrezzi per i veicoli che si suppone
  essere guasti
\item Disporre di un meccanismi di comunicazione broadcast per poter comunicare
  con le auto nelle vicinanze
\item Disporre di un meccasnismo di comunicazione unicast che permetta di
  comunicare con una singola auto
\item Essere in grado di coordinarsi con gli altri veicoli per attraversare
  l'incrocio
\end{itemize}

\subsection{Ambiente}

L'ambiete è l'unico attore che dispone della conoscenza globale dell'incrocio,
compresa la posizione esatta di ogni auto. Deve quindi soddisfare i seguenti
requisiti:
\begin{itemize}
\item Conoscere la posizione esatta di ogni veicoli
\item Rispondere alle richieste dei sensori di prossimità dei veicoli
\item Simulare la comunicazione broadcast inoltrando il messaggio ai soli
  veicoli a portata wireless del mittente
\item Simulare l'intervento del carro attrezzi a seguito di una richiesta di
  intervento da parte di un veicolo
\end{itemize}

La struttara dell'incrocio oltre che fornire una descrizione topologica e
spaziale deve anche indicare le traittorie di percorrenza ammesse. Tale
struttura la si considera come dato di input della simulazione.

\subsection{Interfaccia grafica}

La soluzione prevede la presenza di un interfaccia grafica dalla quale è
possibile visualizzare ad ongi istante lo stato dell'incrocio per monitorarne il
funzionamento e aggiungere nuovi veicoli alla simulazione.


\section{Requisiti non funzionali}
I requisiti non funzionali che una soluzion deve soddisfare sono:
\begin{description}
\item[Sicurezza stradale] Le auto che attraversano l'incrocio non devono
  scontrarsi.
\item[Liveness] Ogni macchina che si trova ad un incrocio deve prima o poi
  attraversarlo.
\item[Fairness] Quando un'auto chiede di impegnare l'incrocio, esiste un tempo
  limite entro cui le viene assegnato.
\item[Deadlock freedom] Non possono verificarsi situazioni in cui nessuna
  macchina nell'incrocio può muoversi.
\item[Near real-time] È necessario imporre un limite di tempo all'esecuzione
  della fase di coordinamento, che deve avvenire in \emph{near real-time} per
  scongiurare fermate prolungate ai veicoli.
\end{description}

Inoltre la soluzione soddisfa le seguenti trasparenza:
\begin{description}
\item[Access transparency] Non c'è un concetto di risorsa remota a cui le
  macchine accedono. Le macchine utilizzano dati ricavati da sensori oppure
  messaggi ricevuti da altre macchine.
\item[Location transparency] Una macchina comunica con i vicini senza conoscerne
  la loro posizione fisica nell'ambiente.
\item[Concurrency transparency] Più macchine possono impegnare l'incrocio senza
  interferire l'una con l'altra.
\item[Failure transparency] Sono ammessi fallimenti fisici e logici delle
  macchine in un qualsiasi momento. Se la macchina si guasta fisicamente è lei
  stessa a ``chiamare il carro attrezzi''; se la macchina si guasta logicamente,
  è la macchina dietro che, passato un certo intervallo di tempo, chiede
  l'intervento di un carro attrezzi. Dovesse guastarsi il leader dell'incrocio,
  si procede con l'elezione di un nuovo leader.
\item[Mobility transparency] Il coordinamento funziona a prescindere dalla
  posizione del leader, che può essere uno qualsiasi fra i veicoli presenti.
\end{description}


\newpage

Come primo passaggio viene creato un processo "environment" dove viene salvato
lo stato globale dell'incrocio. L'incrocio viene tradotto in un grafo orientato
pesato, dove ogni nodo conterrà le seguenti informazioni:
\begin{itemize}
\item Nome
\item Tipologia di nodo: TailNode(Nodo in coda) TopNode (Nodo in prossimità di
  un incrocio) CrossNode(Nodo incrocio)
\item Elenco di macchine che occupano il nodo stesso
\end{itemize}
Il peso degli archi del grafo viene assegnato in base alla distanza fisica tra i
due nodi.
Questo modulo viene interrogato dalle varie macchine per simulare dati relativi
alla sensoristica.

Ogni veicolo sarà implementato come un processo che quindi potrà interrogare i
sensori per ottenere le informazioni su posizione, ostacoli, percorso più breve
verso la sua destinazione.

Le macchine sono in relazione client-server con l'environment. Ogni automobile è
una macchina a stati finiti per la quale si prevede di utilizzare il behaviour
di erlang gen\_statem. Gli stati principali per un veicolo sono:
\begin{itemize}
\item \texttt{INQUEQUE}: La macchina si trova in questo stato quando è in coda
  (non è la prima macchina della fila)
\item \texttt{DISCOVER}: La macchina è la prima della file. In questo stato la
  macchina manda un broadcast per scoprire le auto vicine.
\item \texttt{ELECTION}: La macchina è la prima della fila. E' in corso
  l'elezione di un leader dell'incrocio
\item \texttt{SLAVE}: L'elezione si è conclusa e la macchina non è leader. Mando
  il mio percorso al leader e attendo risposta "verde"/"rosso".
\item \texttt{M-FETCH}: La macchina è stata eletta leader dell'incrocio. Attendo
  i percorsi delle macchine SLAVE.
\item \texttt{M-COORD}: In base ai percorsi e alle priorità delle macchine mando
  messaggi "verde"/"rosso" agli SLAVE.
\item \texttt{CROSSING}: La macchina è in questo stato se si trova su un nodo
  del grafo di tipo CrossNode.
\item \texttt{PH-FAULT}: La macchina ha un guasto fisico.
\end{itemize}

Ogni veicolo può ricevere uno dei seguenti eventi:
\begin{itemize}
\item \texttt{disc}: La macchina è appena arrivata in prossimità dell'incrocio e
  manda in broadcast un "discover" con il suo pid e il suo percorso. A livello
  pratico per simulare il broadcast il messaggio di discover viene mandato
  all'environment che poi rigira alle altre macchine. Queste quindi a loro volta
  mandano un messaggio di "hellofrom" alla nuova macchina (con il loro
  percorso).
\item \texttt{disc\_tm}: lo stato \texttt{DISCOVER} ha un suo timeout. Una volta
  trascorso viene innescato questo evento.
\item \texttt{wait(tag:*)}: se la macchina si trova in stato \texttt{DISCOVER} e
  riceve questo evento rimane nello stato \texttt{DISCOVER} e resetta il
  timeout.
\item \texttt{hello(pid)}: se la macchina si trova in stato \texttt{DISCOVER},
  mi salvo il pid ricevuto tra i conosciuti.
\item \texttt{elect}: Messaggio di \emph{election} dell'algoritmo bully.
\item \texttt{coord}: Messaggio di \emph{coordinator} dell'algoritmo bully.
\item \texttt{ans}: Messaggio di \emph{answer} dell'algoritmo bully.
\item \texttt{elect\_tm}: Evento di timeout (elezione fallita) dell'algoritmo bully.
\item \texttt{route}: messaggio contenente id, priorità, lista macchine con cui
  è in conflitto.
\item \texttt{route\_tm}: evento di timeout per lo stato \texttt{M-FETCH}.
\item \texttt{green}: Messaggio che notifica a macchina in stato \texttt{SLAVE}
  che può attraversare incrocio.
\item \texttt{red}: Messaggio che notifica a macchina in stato \texttt{SLAVE}
  che non può attraversare incrocio.
\item \texttt{gr\_timeout}: evento di timeout per lo stato \texttt{SLAVE} e
  \texttt{M-COORD}.
\end{itemize}

\begin{figure}
  \centering
  \begin{tikzpicture}[shorten >=1pt, node distance=2.5cm,on grid,auto, bend
    angle=30, font=\footnotesize\ttfamily]
    \node[state,initial] (queue) {INQUEUE};%
    \node[state] (disc) [right=of queue] {DISCOVER};%
    \node[state] (elec) [right=of disc,xshift=0.5cm] {ELECTION};%
    \node[state] (slave) [xshift=1cm,above right=of elec] {SLAVE};%
    \node[state] (m-fetch) [xshift=1cm,below right=of elec] {M-FETCH};%
    \node[state] (m-coord) [right=of m-fetch,xshift=0.5cm] {M-COORD};%
    \node[state] (cross) [above right=of m-coord] {CROSSING};%
    \node[state,initial,initial text=fault] (fault) [below=of disc]
    {PH-FAULT};%
    \path[->]%
    (queue) edge node {on\_top} (disc)%
    (disc) edge [loop above] node[xshift=-0.2cm] {disc,hello,wait} ()%
    edge node {disc\_tm} (elec)%
    edge [bend right] node[yshift=-0.5cm] {elec} (elec)%
    edge [bend left] node[yshift=0.1cm,xshift=0.7cm] {coord} (slave)%
    (elec) edge [loop below] node {disc,elect,ans} ()%
    edge node {coord} (slave)%
    edge [bend right] node[yshift=0.45cm] {elec\_tm} (disc)%
    edge node[xshift=-0.4cm,yshift=0.2cm] {master} (m-fetch)%
    (slave) edge [loop
    above] node {disc} () edge node {green} (cross) edge [bend right=50]
          node[xshift=-0.5cm,yshift=0.7cm] {red,gr\_tm} (disc) (m-fetch) edge
          [loop above] node {route,disc} () edge node {route\_tm} (m-coord)
          (m-coord) edge [loop above] node {disc,route} () edge [below] node
          [xshift=0.5cm] {green} (cross) edge [bend left=50] node {red,gr\_tm}
          (disc) (cross) edge [loop above] node {disc} ();
 \end{tikzpicture}
 \caption{Car: Finite State Machine}
\end{figure}

Ogni veicolo ha un suo valore di priorità iniziale, che viene incrementato ogni
volta che il veicolo riceve il messaggio \texttt{red} (lo può ricevere solo se è
in stato \texttt{SLAVE} o \texttt{M-COORD}). Il leader dell'incrocio decide chi
far passare nel seguente modo:
\begin{itemize}
\item Ordina le macchine per priorità;
\item Farà sicuramente passare la prima.
\item Calcola un insieme massimale di macchine indipendenti rispetto alla prima
\end{itemize}

\chapter{Progetto}

\section{Architettura Logica}
Le componenti principali della soluzione sono:
\begin{itemize}
\item Modulo Veicolo: utilizzato per rappresentare il singolo veicolo;
\item Modulo Environment: utilizzato per rappresentare l'ambiente e simulare
  interazione del veicolo con la sensoristica;
\item Interfaccia grafica: modulo utilizzato per monitorare e interagire con l'ambiente
\end{itemize}

Di seguito vengono descritte nel dettaglio le funzionalità implementate da ogni componente
\subsection{Veicolo}
Ogni veicolo è una macchina a stati finiti il cui output dipende dallo stato
attuale e dagli eventi in input. È naturale, quindi, descrivere un veicolo come
una macchia di Mealy, in cui gli input sono generici eventi e gli output sono
messaggi inviati. Gli stati principali per un veicolo sono:
\begin{itemize}
\item \texttt{INQUEUE}: Il veicolo si trova in questo stato quando stà
  transitando in una zona non in prossimità di un incrocio;
\item \texttt{DISCOVER}: Il veicolo si trova in prossimità di un incrocio e deve
  effettuare il \emph{discover} delle auto nelle vicinanze
\item \texttt{WAIT}: Il veicolo deve attendere il completamento di un elezione
  già in corso o il transito di altri veicoli nell'incrocio.
\item \texttt{ELECTION}: Il veicolo si trova in prossimità di un incrocio. E' in
  corso l'elezione di un leader dell'incrocio
\item \texttt{SLAVE}: L'elezione si è conclusa e il veicolo non è leader, quindi
  resta in attesa di istruzioni da parte del leader;
\item \texttt{MASTER}: Il veicolo è stato eletto leader dell'incrocio, e quindi
  procede inviando istruzioni ai veicoli SLAVE;
\item \texttt{CROSSING}: Il veicolo stà impegnando l'incrocio;
\item \texttt{PH-FAULT}: Il veicolo ha rilevato un guasto fisico. Un guasto
  fisico può avvenire in un qualsiasi momento.
\end{itemize}

Per ogni stato si elencano gli eventi che il veicolo è tenuto a gestire, come ad
esempio messaggi ricevuti o timer interni, e le azioni intraprese allo scaturire
di tali eventi. Gli eventi non menzionati in un certo stato vanno ignorati.

\paragraph{\texttt{INQUEUE}}
Il veicolo avanza di una cella nello spazio fisico ogni \emph{n} secondi
seguendo la sua rotta e rallentando nel caso in cui il sensore di prossimità
rilevi un ostacolo oppure si arrivi in prossimità dell'incrocio.

\begin{figure}[h]
  \centering
  \begin{tikzpicture}[shorten >=1pt, node distance=2.5cm,on grid,auto, bend
    angle=30, font=\footnotesize\ttfamily]
    \node[state,initial, initial text=enter/\_] (queue) {INQUEUE};%
    \node[state] (disc) [right=of queue, xshift=1cm] {DISCOVER};%
    \path[->] (queue) edge [loop above] node[align=center]%
    {car\_mv/\_\\fault/no} ()%
                      edge node {car\_mv/\_} (disc);%
  \end{tikzpicture}
\end{figure}

L'evento \texttt{car\_mv} rappresenta l'avanzamento del veicolo lungo la sua
rotta. Alla ricezioni di questo evento il veicolo interroga il sensore di
prossimità: se vi sono ostacoli, il veicolo si ferma in attesa di un nuovo
evento \texttt{car\_mv}; se la strada è libera procede con la sua corsa fino ad
occupare la cella successiva nella griglia. A questo punto, se la nuova cella
risulta essere in prossimità di un incrocio, il veicolo entra nello stato
\texttt{DISCOVER}, altrimenti rimane nello stesso stato in attesa di un nuovo
evento \texttt{car\_mv}.

Se il veicolo rileva un ostacolo che lo obbliga a una fermata prolungata, avvia
una chiamata al soccorso stradale in merito alla posizione dell'ostacolo. Sarà
poi il soccorso stradale a rimuovere l'eventuale veicolo dalla posizione
segnalata, se effettivamente guasto. Il soccorso stradale è una funzionalità
offerta dal modulo \emph{Environment}.

\paragraph{\texttt{DISCOVER}}
In prossimità di un incrocio il veicolo deve capire se vi siano altri veicoli in
prossimità dello stesso incrocio. Viene quindi applicato il protocollo di
\emph{discovery} descritto nella \cref{sec:protalg}.

\begin{figure}[h]
  \centering
  \begin{tikzpicture}[shorten >=1pt, node distance=2.5cm,on grid,auto, bend
    angle=30, font=\footnotesize\ttfamily]
    \node[state, initial, initial text=enter/disc] (disc) {DISCOVER};%
    \node[state] (elec) [above right=of disc,xshift=2.5cm] {ELECTION};%
    \node[state] (slave) [xshift=2.5cm,below right=of disc] {SLAVE};%
    \node[state] (wait) [below=of disc] {WAIT};%
    \path[->] (disc) edge [loop above] node[align=center]
    {disc/hello\\hello/\_\\fault/no} ()%
    edge node[align=center, xshift=0.7cm,yshift=0.2cm] {disc\_tm/\_\\elect/\_} (elec)%
    edge node {wait/\_} (wait)%
    edge node {coord/\_} (slave);%
  \end{tikzpicture}
\end{figure}
Quando il veicolo entra nello stato \texttt{DISCOVER} invia il messaggio
broadcast \texttt{disc}, che verrà ricevuto dai soli veicoli a portata wireless,
contenente la propria identità e la propria rotta. Rimane poi in attesa dei
seguenti messaggi fino allo scadere di un timeout (evento \texttt{disc\_tm}):
\begin{itemize}
\item \texttt{disc}: Un altro veicolo ha segnalato la sua presenza
  nell'incrocio. Il veicolo memorizza le informazioni sul mittente e risponde
  con un messaggio \texttt{hello} per segnalare la propria presenza.
\item \texttt{hello}: Ricevuto in risposta a un \texttt{disc}; il veicolo
  memorizza le informazioni sul mittente.
\item \texttt{wait}: Ricevuto in risposta a un \texttt{disc} quando un elezione
  è già in corso o un veicolo sta impegnando l'incrocio. Il veicolo entra nello
  stato \texttt{WAIT} per un breve periodo prima di ritornare nella fase di
  \emph{discovery}.
\item \texttt{elect}: Un altro veicolo ha indetto un'elezione. Il veicolo entra
  nello stato \texttt{ELECTION} per partecipare all'elezione.
\item \texttt{coord}: Un altro veicolo è stato eletto il leader dell'incrocio.
  Il veicolo memorizza l'identità del leader e entra nello stato \texttt{SLAVE}.
\item \texttt{fault}: Il soccorso stradale ha richiesto lo stato del veicolo.
  Non essendoci anomalie, viene inviata una risposta negativa.
\end{itemize}

Allo scadere del timeout \texttt{disc\_tm} il veicolo entra nello stato
\texttt{ELECTION} e indice un'elezione con i veicoli di cui conosce l'identità.

\paragraph{\texttt{WAIT}}
Il veicolo entra in questo stato a seguito della ricezione di un messaggio
\texttt{wait} e vi rimane fino allo scadere del timeout \texttt{wait\_tm}.
Eventuali messaggi ricevuti in questo stato vengo ignorati.

\begin{figure}[h]
  \centering
  \begin{tikzpicture}[shorten >=1pt, node distance=2.5cm,on grid,auto, bend
    angle=30, font=\footnotesize\ttfamily]
    \node[state] (disc) {DISCOVER};%
    \node[state] (wait) [below=of disc] {WAIT};%
    \path[->] (disc) edge [bend left] node {wait/\_} (wait)%
              (wait) edge [bend left] node {wait\_tm/\_} (disc);
  \end{tikzpicture}
\end{figure}

\paragraph{\texttt{ELECTION}}
Quando il veicolo entra in questo stato indice un elezione secondo l'algoritmo
\emph{bully}: invia un messaggio \texttt{elect} ai soli veicoli con un
identificativo maggiore del proprio e attende risposta. Il veicolo con
l'identificativo più alto si elegge leader (evento \texttt{master}), entrando
quindi nello stato \texttt{MASTER} prima di inviare il messaggio \texttt{coord}
agli altri partecipanti.

\begin{figure}[h]
  \centering
  \begin{tikzpicture}[shorten >=1pt, node distance=2.5cm,on grid,auto, bend
    angle=30, font=\footnotesize\ttfamily]
    \node[state, initial, initial text=enter/elect] (elec) {ELECTION};%
    \node[state] (slave) [above right=of elec, xshift=2.5cm] {SLAVE};%
    \node[state] (master) [xshift=2.5cm,below right=of elec] {MASTER};%
    \node[state] (disc) [below=of elec] {DISCOVER};%
    \path[->] (elec) edge [loop above] node[align=center]
    {elect/ans\\ans/\_\\disc/wait\\fault/no} ()%
    edge node {coord/\_} (slave)
    edge node {master/\_} (master)
    edge node {elect\_tm/\_} (disc);
  \end{tikzpicture}
\end{figure}

Il veicolo in questo stato gestisce i seguenti eventi:
\begin{itemize}
\item \texttt{elect}: Corrisponde al messaggio \emph{election} dell'algoritmo
  \emph{bully}; il veicolo è tenuto a rispondere con un messaggio \texttt{ans}.
\item \texttt{ans}: Corrisponde al messaggio \emph{answer} dell'algoritmo
  \emph{bully}.
\item \texttt{coord}: Corrisponde al messaggio \emph{coordinator} dell'algoritmo
  \emph{bully}; il veicolo entra nello stato \texttt{SLAVE}.
\item \texttt{elect\_tm}: Il veicolo non ha ricevuto il messaggio \texttt{coord}
  entro il tempo prestabilito, quindi assume che l'elezione sia fallita e
  ritorna nello stato \texttt{DISCOVER}.
\item \texttt{master}: Il veicolo si elegge leader e entra nello stato
  \texttt{MASTER}.
\item \texttt{disc}: Un nuovo veicolo è giunto in prossimità dell'incrocio.
  Essendoci un elezione in corso, il veicolo invia un \texttt{wait} in risposta.
\item \texttt{fault}: Il soccorso stradale ha richiesto lo stato del veicolo.
  Non essendoci anomalie, viene inviata una risposta negativa.
\end{itemize}

\paragraph{\texttt{SLAVE}}
Il veicolo, conoscendo le rotte degli altri partecipanti, calcola con quali di
esse la sua rotta è in conflitto e lo comunica al leader tramite il messaggio
\texttt{conflict}. Quindi rimane in attesa di istruzioni da parte del leader per
un tempo prestabilito, scaduto il quale (evento \texttt{slave\_tm}) lo assume
guasto e ritorna nello stato \texttt{DISCOVER}. In questo stato il veicolo
gestisce i seguenti eventi:
\begin{itemize}
\item \texttt{green}: Il veicolo può attraversare l'incrocio, quindi entra nello
  stato \texttt{CROSSING}.
\item \texttt{red}: Il veicolo non è autorizzato ad impegnare l'incrocio, quindi
  aumenta la sua priorità e ritorna nello stato \texttt{DISCOVER}.
\item \texttt{elect}: Il veicolo inoltra un \texttt{ans} al mittente ancora
  ignaro del leader.
\item \texttt{disc}: Un nuovo veicolo è giunto in prossimità dell'incrocio. Non
  avendo partecipato all'elezione ormai conclusasi, il veicolo invia un
  \texttt{wait} in risposta.
\item \texttt{fault}: Il soccorso stradale ha richiesto lo stato del veicolo.
  Non essendoci anomalie, viene inviata una risposta negativa.
\end{itemize}


\begin{figure}[h]
  \centering
  \begin{tikzpicture}[shorten >=1pt, node distance=2.5cm,on grid,auto, bend
    angle=30, font=\footnotesize\ttfamily]
    \node[state,initial, initial text=enter/conflict] (slave) {SLAVE};%
    \node[state] (cross) [xshift=2.5cm,above right=of slave] {CROSSING};%
    \node[state] (disc) [xshift=2.5cm,below right=of slave] {DISCOVER};%
    \path[->] (slave) edge [loop above] node[align=center]
    {elect/ans\\disc/wait\\fault/no} ()%
    edge node [xshift=0.6cm, yshift=0.2cm]{green/\_} (cross)
    edge node [align=center, xshift=-0.5cm, yshift=0.2cm]{red/\_\\slave\_tm/\_} (disc);
  \end{tikzpicture}
\end{figure}

\paragraph{\texttt{MASTER}}
Il veicolo calcola con quali partecipanti è in conflitto e rimane in attesa dei
messaggi \texttt{conflict} fino allo scadere di un timeout (evento
\texttt{master\_tm}). Ricevuto l'evento \texttt{master\_tm}, procede al calcolo
di un insieme massimale di veicoli le cui rotte non sono in conflitto e invia
messaggi \texttt{green} e \texttt{red} di conseguenza. In questo stato il
veicolo gestisce i seguenti eventi:
\begin{itemize}
\item \texttt{green}: Il veicolo può attraversare l'incrocio, quindi entra nello
  stato \texttt{CROSSING}.
\item \texttt{red}: Il veicolo non è autorizzato ad impegnare l'incrocio, quindi
  aumenta la sua priorità e ritorna nello stato \texttt{DISCOVER}.
\item \texttt{elect}: Il veicolo inoltra un \texttt{ans} al mittente ancora
  ignaro del leader.
\item \texttt{disc}: Un nuovo veicolo è giunto in prossimità dell'incrocio. Non
  avendo partecipato all'elezione ormai conclusasi, il veicolo invia un
  \texttt{wait} in risposta.
\item \texttt{fault}: Il soccorso stradale ha richiesto lo stato del veicolo.
  Non essendoci anomalie, viene inviata una risposta negativa.
\end{itemize}


\begin{figure}[h]
  \centering
  \begin{tikzpicture}[shorten >=1pt, node distance=2.5cm,on grid,auto, bend
    angle=30, font=\footnotesize\ttfamily]
    \node[state,initial, initial text=enter/\_] (master) {MASTER};%
    \node[state] (cross) [xshift=2.5cm,above right=of master] {CROSSING};%
    \node[state] (disc) [xshift=2.5cm,below right=of master] {DISCOVER};%
    \path[->] (master) edge [loop above] node[align=center]
    {conflict/\_\\master\_tm/\{green,red\}\\elect/ans\\disc/wait\\fault/no} ()%
    edge node [xshift=0.6cm, yshift=0.2cm]{green/\_} (cross)
    edge node [xshift=-0.5cm, yshift=0.2cm]{red/\_} (disc);
  \end{tikzpicture}
\end{figure}

\paragraph{\texttt{CROSSING}} Il veicolo impegna l'incrocio secondo la sua rotta
e con velocità costante fino a liberarlo. L'evento \texttt{car\_mv} scandisce il
movimento del veicolo: alla sua ricezione il veicolo interroga i sensori di
prossimità, per poi avanzare alla cella successiva in assenza di ostacoli
nell'incrocio. Lasciata alle spalle l'ultima cella dell'incrocio, il veicolo
ritorna nello stato \texttt{INQUEUE}. Agli eventi \texttt{disc} il veicolo è
tenuto a rispondere con un messaggio di \texttt{wait} fintanto che si trova in
questo stato.


\begin{figure}[h]
  \centering
  \begin{tikzpicture}[shorten >=1pt, node distance=2.5cm,on grid,auto, bend
    angle=30, font=\footnotesize\ttfamily]
    \node[state,initial, initial text=enter/\_] (cross) [xshift=1cm] {CROSSING};%
    \node[right=of cross,state,xshift=1cm] (queue) {INQUEUE};%
    \path[->] (cross) edge [loop above] node[align=center]%
    {car\_mv/\_\\disc/wait\\fault/no} ()%
    edge node {car\_mv/\_} (queue);%
  \end{tikzpicture}
\end{figure}


\paragraph{\texttt{PH-FAULT}}
Ad ogni istante il veicolo può subire un guasto fisico e quindi doversi fermare
in attesa che il soccorso stradale intervenga. Un guasto visico viene segnalato
dall'evento \texttt{ph\_fault} che obbliga il veicolo ad entrare nello stato
\texttt{PH-FAULT}.


\subsection{Environment}

Si riportano di seguito le API pubbliche dell'ambiente che è possibile
richiamare dall'esterno:
\begin{itemize}
\item \texttt{broadcast\_disc}: funzione richiamata da un veicolo per simulare
  un messaggio broadcast. La funzione calcola le auto che si trovano a portata
  wireless del veicolo e quindi inoltra il messaggio.
\item \texttt{req\_prox\_sensor\_data}: funzione utilizzata dal veicolo per
  simulare una query al sensore di prossimità. La funzione verificare se sia
  presente o meno un veicolo nella posizione richiesta, quindi inoltra l'esito
  al veicolo. Nota: non viene controllato lo stato del veicolo che potrebbe
  essere quindi anche in guasto fisico o logico.
\item \texttt{notify\_move}: funzione utilizzata dal veicolo per avvisare
  l'ambiente che si è mosso. La funzione provvede ad aggiornare lo stato globale
  dell’incrocio. Una volta concluse tutte le operazioni viene inviato il nuovo
  stato globale all’interfaccia grafica.
\item \texttt{spawn\_car}: funzione utilizzata per inserire un nuovo veicolo
  nella simulazione. Viene utilizzata principalmente dall'interfaccia grafica.
  Una volta concluse tutte le operazioni viene inviato il nuovo stato globale
  all’interfaccia grafica.
\item \texttt{delete\_car}: la funzione cancella un veicolo dall'ambiente. Viene
  utilizzata dal carro attrezzi in caso di veicoli guasti oppure dal veicolo
  stesso una volta che raggiunge la sua destinazione nell’ambiente. Una volta
  concluse tutte le operazioni viene inviato il nuovo stato globale
  all’interfaccia grafica.
\item \texttt{request\_towtruck}: funzione utilizzata dai veicoli per richiedere
  l'intervento del carro attrezzi. La funzione verifica se il veicolo segnalato
  sia effettivamente guasto, e in caso positivo provvede a rimuoverlo
  dall’ambiente. I guasti logici vengono rilevati mediante un timeout: se il
  veicolo non risponde alla richiesta entro un determinato intervallo di tempo
  viene considerato in guasto logico. Una volta concluse tutte le operazioni
  viene inviato il nuovo stato globale all’interfaccia grafica.
\end{itemize}

\subsection{Interfaccia grafica}

Si riportano di seguito i messaggi che è possibile inviare all'interfaccia grafica:
\begin{itemize}
\item \texttt{globalstate}: Alla ricezione di questo messaggio l’interfaccia
  grafica procede aggiornando la visualizzazione dell’incrocio secondo il nuovo
  stato ricevuto.
\item \texttt{cardata}: Viene invocata dal veicolo quando entra in un nuovo
  stato così da mostrare l'evento anche dall'interfaccia grafica.
\item \texttt{carcolor}: Viene invocata dal veicolo quando questo riceve un
  messaggio \texttt{green}/\texttt{red} oppure entra in \texttt{PH-FAULT}. Alla
  ricezione di tale messaggio l’interfaccia va a modificare il colore associato
  al veicolo.
\end{itemize}



\newpage
This chapter is devoted to the description of the general architectures, and specific algorithms.

\section{Logical architecture}
Describe the components of your systems: modules/objects/components/services.
For each component, describe the functionalities it implements, and by who is used.


\section{Protocols and algorithms}
\label{sec:protalg}
Communication between components.  UML sequence diagrams go here.

Also, put here a detailed description of distributed algorithms used to solve specific problems of the project.

\section{Physical architecture and deployment}
Which nodes and platforms involved, and where each component is deployed.

\section{Development plan}
Since it is difficult to predict just how hard implementing a new system will be, you should formulate as a set of ``tiers,'' where the basic tier is something you?re sure you can complete, and the additional tiers add more features, at both the application and the system level.

\chapter{Implementation}

Details about the implementation: every choice about platforms, languages, software/hardware, middlewares, which has not been decided in the requirements.


Important choices about implementation should be described here; e.g., peculiar data structures.


\chapter{Validation}

Check if requirements from Chapter~\ref{ch:analysis} have been fulfilled.
Quantitative tests (simulations) and screenshots of the interfaces are put here.


\chapter{Conclusions}

What has been done with respect to what has been promised in Chapters~\ref{ch:intro} and \ref{ch:analysis}, and what is left out.

\appendix

\chapter{Appendix}

In the Appendix you can put code snippets, snapshots, installation instructions, etc.


\chapter*{Evaluation}
Your system will be judged mainly on how it operates as a distributed system. The primary evaluation will be according to whether your system has the following attributes:
\begin{itemize}
\item  It should be an interesting distributed system, making use of some of the algorithms we have covered in class for distributed synchronization, replication, fault tolerance and recovery, security, etc.
\item The software should be well designed and well implemented, in terms of the overall architecture and the detailed realization.
\item You should devise and apply systematic testing procedures, at both the unit and systems levels.
\item The system should operate reliably and with good performance, even in the face of failures.
\end{itemize}
Important, but secondary considerations include:
\begin{itemize}
\item Time taken to do the project (the sooner the better, but do not miss details in order to end sooner)
\item  How nice is the application's appearance: does it have a nice interface or a compelling visual display?
\end{itemize}

\end{document}


% Local Variables:
% ispell-dictionary: "italiano"
% End:
